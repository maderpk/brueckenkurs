\documentclass[notheorems]{beamer}

% BEAMER SETTINGS
\usetheme{CambridgeUS}
\usecolortheme{dolphin}

\setbeamertemplate{theorems}[numbered]

\AtBeginSection[]
{
    \begin{frame}{Agenda}
        \tableofcontents[currentsection]
    \end{frame}
}

\AtBeginSubsection[]
{
    \begin{frame}{Agenda}
        \tableofcontents[currentsection,currentsubsection]
    \end{frame}
}


% IMPORTS
\usepackage[ngerman]{babel}
\usepackage[utf8x]{inputenc}
\usepackage{amsfonts}
\usepackage{amsmath}
\usepackage{amssymb}
\usepackage{amstext}
\usepackage{float}
\usepackage{hyperref}

% NEWCOMMANDS
% Syntax: \newcommand{\shortcut}{\wasrauskommensoll}

% Shortcuts für Zahlenbereiche:
\newcommand{\N}{\mathbb{N}}
\newcommand{\R}{\mathbb{R}}
\newcommand{\Z}{\mathbb{Z}}
\newcommand{\C}{\mathbb{C}}
\newcommand{\Q}{\mathbb{Q}}

% THEOREMSTYLES

\newtheorem{theorem}{Satz} % Sätze, sollte später evtl. noch Lemma und Korollar hinzufügen
\numberwithin{theorem}{section}

\newtheorem{mydef}[theorem]{Definition} % Definitionen
\numberwithin{theorem}{section}

\newtheorem{example}[theorem]{Beispiel} % Beispiele
\numberwithin{theorem}{section}

\newtheorem{proposition}[theorem]{Proposition} % Propositionen
\numberwithin{theorem}{section}

\newtheorem{remark}[theorem]{Bemerkung} % Bemerkungen
\numberwithin{theorem}{section}

\newtheorem{lemma}[theorem]{Lemma} % Lemmata
\numberwithin{theorem}{section}

\newtheorem*{proof*}{Beweis:}

% PDF-META
\hypersetup{pdftex,
            pdfauthor={Jonas Köppl, Maximilian Reif, Peter Mader},
            pdftitle={Brückenkurs Mathematik -- Beamer-Präsentation},
            pdfsubject={},
            pdfkeywords={},
            pdfproducer={},
            pdfcreator={},
            pdfpagemode=UseOutlines
}

\title{Brückenkurs Mathematik}
\author[]{Jonas Köppl, Maximilian Reif, Peter Mader}
\institute[]{Universität Passau}
\date{}

\begin{document}

\begin{frame}
    \titlepage
\end{frame}

\begin{frame}{Agenda}
    \tableofcontents
\end{frame}

\section{Einfuehrung}

\section{Logische Grundlagen}

\begin{frame}
Bevor wir uns mit Beweisen beschäftigen können, müssen zunächst die dafür
notwendigen logischen Grundlagen erarbeitet werden.
Hierzu benötigen wir eingangs ein paar Definitionen.
\end{frame}


\begin{frame}
\begin{mydef}
Eine \textit{Aussage} ist ein sprachliches Gebilde, das entweder wahr oder
falsch ist.
Dabei ist es nicht erforderlich, sagen zu können, \textit{ob} die Aussage wahr
oder falsch ist.
\end{mydef}
\end{frame}


\begin{frame}
Am besten veranschaulichen wir uns das mit einem Beispiel und einer Übungsaufgabe.

\begin{example}
\begin{enumerate}
\item Es regnet. (Zeit- und ortsabhängig wahr oder falsch.)
\item 11 ist durch 5 teilbar. (Falsch.)
\item Es gibt unendlich viele Primzahlen. (Wahr, Beweis folgt.)
\item Es gibt unendlich viele Primzahlzwillinge. (Unbewiesen, Stand 17.10.18.)
\end{enumerate}
\end{example}
\end{frame}


\begin{frame}
Wir interessieren uns vor allem für mathematische Aussagen und deren
Wahrheitsgehalt.
Zunächst benötigen wir jedoch noch Möglichkeiten, um mehrere Aussagen
\textit{logisch zu verknüpfen}.
\end{frame}


\begin{frame}
\begin{mydef}
Wenn $P$ und $Q$ Aussagen sind, dann heißt $P \wedge Q$ die
\textit{Konjunktion} von $P$ und $Q$.
Der Wahrheitswert von $P \wedge Q$ ist definiert durch die Wahrheitswerte von
$P$ und $Q$ mittels folgender \textit{Wahrheitstafel}:

\begin{table}[H]
\centering
\begin{tabular}{c|c|c}
$P$ & $Q$ & $P \wedge Q$ \\ \hline
w   & w   & w \\
w   & f   & f \\
f   & w   & f \\
f   & f   & f
\end{tabular}
\end{table}

Das heißt $P \wedge Q$ ist genau dann wahr, wenn $P$ und $Q$ beide wahr sind
und sonst falsch.
\end{mydef}
\end{frame}


\begin{frame}
\begin{example}
\begin{enumerate}
\item ($\sqrt{2}$ ist irrational) $\wedge$ ($\sqrt{2} > 0$) (Wahr.)
\item $(2 + 2 = 4) \wedge (3 + 2 = 7)$ (Falsch.)
\end{enumerate}
\end{example}
\end{frame}


\begin{frame}
\begin{mydef}
Wenn $P$ und $Q$ Aussagen sind, so heißt $P \vee Q$ die \textit{Disjunktion}
von $P$ und $Q$.
Die definierende Wahrheitstafel ist gegeben durch:

\begin{table}[H]
\centering
\begin{tabular}{c|c|c}
$P$ & $Q$ & $P \vee Q$ \\ \hline
w   & w   & w \\
w   & f   & w \\
f   & w   & w \\
f   & f   & f
\end{tabular}
\end{table}

Das heißt $P \vee Q$ ist genau dann wahr, wenn mindestens eine der beiden
Teilaussagen wahr ist.
\end{mydef}
\end{frame}


\begin{frame}
\begin{example}
\begin{enumerate}
\item ($\sqrt{2}$ ist irrational) $\vee$ ($\sqrt{2} > 0$) (Wahr.)
\item $(2 + 2 = 4) \vee (3 + 2 = 7)$ (Wahr.)
\end{enumerate}
\end{example}

\begin{remark}
Im Gegensatz zur Umgangssprache ist mit dem mathematischen oder stets das
\textit{inklusive oder} gemeint.
Möchte man das \textit{exklusive oder} verwenden, so nutzt man den Ausdruck
\glqq entweder \ldots\ oder\ldots\grqq.
\end{remark}
\end{frame}


\begin{frame}
\begin{mydef}
Wenn $P$ eine Aussage ist, dann heißt $\neg P$ die \textit{Negation} von $P$.
Definierende Wahrheitstafel:

\begin{table}[H]
\centering
\begin{tabular}{c|c}
$P$ & $\neg P$ \\ \hline
w   & f \\
f   & w
\end{tabular}
\end{table}
\end{mydef}
\end{frame}


\begin{frame}
\begin{mydef}
Wenn $P$ und $Q$ zwei Aussagen sind, so heißt $P \Rightarrow Q$
(sprich: wenn $P$, dann $Q$) die \textit{Implikation} von Q durch P.
Definierende Wahrheitstafel:

\begin{table}[H]
\centering
\begin{tabular}{c|c|c}
$P$ & $Q$ & $P \Rightarrow Q$ \\ \hline
w   & w   & w \\
w   & f   & f \\
f   & w   & w \\
f   & f   & w
\end{tabular}
\end{table}
\end{mydef}

\begin{example}
\begin{enumerate}
\item Wenn $ 3 > 2$, dann teilt $5$ die Zahl $10$. (Wahr.)
\item Wenn $2 > 3$, dann ist die Erde eine Scheibe. (Wahr.)
\end{enumerate}
\end{example}
\end{frame}


\begin{frame}
\begin{mydef}
Wenn $P$ und $Q$ Aussagen sind, so heißt die Verknüpfung
$(P \Rightarrow Q) \wedge (Q \Rightarrow P)$ die \textit{Äquivalenz} von
$P$ und $Q$.
Abkürzend schreibt man auch $P \iff Q$.
Die definierende Wahrheitstafel ist:

\begin{table}[H]
\centering
\begin{tabular}{c|c|c}
$P$ & $Q$ & $P \iff Q$ \\ \hline
w   & w   & w \\
w   & f   & f \\
f   & w   & f \\
f   & f   & w
\end{tabular}
\end{table}
\end{mydef}
\end{frame}

\section{Mengen und Elemente}

\subsection{Mengen und Teilmengen}

\begin{frame}
\begin{mydef}
Eine \textit{Menge} $M$ ist eine Zusammenfassung von bestimmten
wohlunterschiedenen Objekten unserer Anschauung oder unseres Denkens zu einem
Ganzen.
Die in einer Menge zusammengefassten Objekte werden auch als \textit{Elemente}
von $M$ bezeichnet.
\end{mydef}

\begin{remark}
Um besser über Mengen und ihre Elemente sprechen zu können, brauchen wir noch
ein wenig Notation.

\begin{enumerate}
\item Ist $x$ ein Element von $M$, so schreiben wir $x \in M$.
\item Ist $x$ kein Element von $M$, so schreiben wir $x \notin M$.
\end{enumerate}
\end{remark}
\end{frame}


\begin{frame}
\begin{example}
In diesem Beispiel lernen wir einige Mengen und gängige Bezeichnungen kennen,
damit wir uns später Schreibarbeit sparen können.

\begin{enumerate}
\item Die Menge der \textit{natürlichen Zahlen}: $\N = \{1,2,3,\ldots\}$.
\item Die Menge der \textit{natürlichen Zahlen} inklusive der $0$:
$\N_0 = \{0,1,2,3,\ldots\}$.
\item $M = \{ \text{rot, gelb, blau} \}$.
\item Die Menge aller Geraden und Dreiecke der Ebene.
\item Die Menge aller Polynome $a_0 + a_1 x + \ldots + a_n x^n$ mit reellen
Koeffizienten $a_0,a_1,\ldots,a_n$.
\item Die leere Menge $\emptyset$. Sie enthält keine Elemente.
\end{enumerate}
\end{example}
\end{frame}


\begin{frame}
\begin{mydef}
Seien $X,Y$ Mengen.
Die Menge $X$ heißt \textit{Teilmenge} von $Y$ (und $Y$ heißt
\textit{Obermenge} von $X$), falls jedes Element von $X$ auch ein Element
von $Y$ ist.
In Zeichen: $X \subseteq Y$.
Für die Negation von $X \subseteq Y$ wird $X \nsubseteq Y$ geschrieben.
\end{mydef}

\begin{example}
\begin{enumerate}
\item Die Menge $\N$ ist eine Teilmenge von $\N_0$.
\item Es gilt: $\N \subseteq \Z \subseteq \Q \subseteq \R$.
\end{enumerate}
\end{example}
\end{frame}


\begin{frame}
\begin{mydef}
Sei $X$ eine Menge, so bezeichne $\mathcal{P}(X)$ die Menge aller Teilmengen
von $X$.
$\mathcal{P}(X)$ heißt die \textit{Potenzmenge} von $X$.
\end{mydef}

\begin{example}
Betrachte die Menge $X = \{1,2\}$.
Dann gilt:
\begin{align*}
\mathcal{P}(X) = \{\emptyset, \{1\}, \{2\}, \{1,2\}\}
\end{align*}
\end{example}
\end{frame}


\subsection{Prädikate und Erzeugung von Teilmengen}

\begin{frame}
Betrachte zunächst die folgenden beiden Sätze:

\begin{enumerate}
\item[(i)]  Für jede natürliche Zahl $x$ gilt $x \geq 1$.
\item[(ii)] $x$ ist kleiner als 5.
\end{enumerate}

Dann ist (i) eine (wahre) Aussage, während man von (ii) nicht sagen kann,
ob es wahr oder falsch ist.
Also ist (ii) keine Aussage.
Setzt man in (ii) für $x$ eine beliebige natürliche Zahl ein, so erhält man
aber eine Aussage.
Im Satz (i) kommt $x$ als \textit{gebundene} Variable vor, im Satz (ii) als
\textit{freie} bzw. \textit{ungebundene Variable}.


\begin{mydef}
Sätze (hier im umgangssprachlichen Sinn), in denen eine freie Variable $x$
vorkommt und die nach Ersetzen dieser Variablen $x$ durch ein mathematisches
Objekt zu einer Aussage werden, heißen \textit{Prädikate} (= Eigenschaften)
von $x$.
Sie werden beispielsweise mit $P(x)$ bezeichnet.
\end{mydef}
\end{frame}

\begin{frame}
\begin{remark}
Ein Prädikat $P(x)$ ist nicht zu verwechseln mit der Potenzmenge
$\mathcal{P}(X)$.
Letztere hat als Argument immer eine Menge.
\end{remark}

Einige oft verwendete Prädikate erhalten eigene Symbole.
\end{frame}

\begin{frame}
\begin{mydef}
Sei $X$ eine Menge und $P(x)$ ein Prädikat.
\begin{enumerate}
\item Die Aussage \glqq Für alle $x \in X$ gilt $P(x)$\grqq wird abgekürzt
durch:
\[
  \forall x \in X\!:\ P(x).
\]
Das Symbol $\forall$ wird auch als \textit{Allquantor} bezeichnet.

\item Die Aussage \glqq Es gibt ein $x \in X$, für das $P(x)$ gilt\grqq wird
abgekürzt durch:
\[
  \exists x \in X\!:\ P(x).
\]
Das Symbol $\exists$ wird auch als \textit{Existenzquantor} bezeichnet.

\item Die Aussage \glqq Es gibt genau ein $x \in X$, für das $P(x)$ gilt\grqq
wird abgekürzt durch:
\[
  \exists ! x \in X\!:\ P(x).
\]
oder
\[
  \exists_1 x \in X\!:\ P(x).
\]
\end{enumerate}
\end{mydef}
\end{frame}

\begin{frame}
\begin{remark}
Es gilt:
\begin{align*}
  [ \neg (\forall x \in X\!:\ P(x))] \iff [\exists x \in X\!:\ \neg P(x)]. \\\
  [ \neg (\exists x \in X\!:\ P(x))] \iff [\forall x \in X\!:\ \neg P(x)].
\end{align*}
\end{remark}
\end{frame}


\begin{frame}
\begin{mydef}
Sei $X$ eine Menge und $P(x)$ ein Prädikat.
Dann bezeichne
\[
  \{ x \in X \mid P(x) \}
\]
diejenige Teilmenge von $X$, die aus allen Elementen von $X$ besteht,
für die $P(x)$ wahr ist.
Diese Darstellung einer Menge heißt auch \textit{intensional}.
Das bloße Aufzählen der beinhalteten Elemente wird als \textit{extensional}
bezeichnet.
\end{mydef}
\end{frame}

\begin{frame}
\begin{example}
\begin{enumerate}
\item $\{n \in \N \mid \text{n ist gerade} \} = \{2,4,6,\ldots\}$.
\item $\{n \in \Z \mid \text{n ist ungerade} \} = \{\ldots,-3,-1,1,3,\ldots\}$.
\end{enumerate}

Die Beschreibung auf der linken Seite ist jeweils die intensionale Schreibweise,
während auf der rechten Seite die extensionale Mengenschreibweise verwendet wird.
\end{example}
\end{frame}


\begin{frame}
Mit Hilfe der gerade eingeführten Notation können wir nun elementare
Mengenoperationen definieren.

\begin{mydef}
Seien $X,Y$ zwei Mengen.
\begin{enumerate}
\item $X \cup Y := \{x \mid x \in X \vee x \in Y \}$
heißt \textit{Vereinigung} von $X$ und $Y$.

\item $X \cap Y := \{x \mid x \in X \wedge x \in Y \}$
heißt \textit{Durchschnitt} von $X$ und $Y$.

\item $X \setminus Y := \{x \mid x \in X \wedge x \notin Y \}$
heißt \textit{relatives Komplement} von $Y$ in $X$.

\item Ist $Y \subset X$, so heißt $X \setminus Y$
das \textit{Komplement} von $Y$ bzgl. $X$ und wird mit $Y^c$ bezeichnet.
\end{enumerate}
\end{mydef}
\end{frame}


\begin{frame}
\begin{mydef}
Durchschnitte und Vereinigungen von Mengen lassen sich auf beliebig viele
Mengen verallgemeinern.
Sei dazu $I$ eine Menge und für alle $i \in I$ sei $A_i$ eine Menge.

\begin{enumerate}
\item Die Menge
$\bigcup_{i \in I} A_i := \{x \mid \exists i \in I\!:\ x \in A_i \}$
heißt die Vereinigung der Mengen $A_i$, $i \in I$.

\item Falls $I \neq \emptyset$ heißt die Menge
$\bigcap_{i \in I} A_i := \{x \mid \forall i \in I\!:\ x \in A_i \}$
der Durchschnitt der Mengen $A_i$, $i \in I$.
\end{enumerate}
\end{mydef}
\end{frame}


\begin{frame}
\begin{remark}
\begin{enumerate}
\item Schreibweise für $I = \N$: $\bigcap_{i=1}^{\infty}A_i$ bzw.
$\bigcup_{i=1}^{\infty}A_i$.

\item Schreibweise für $I = \{1,...,n\}$: $\bigcap_{i=1}^n A_i$ bzw.
$\bigcup_{i=1}^n A_i$.
\end{enumerate}
\end{remark}
\end{frame}

\section{Aufbau mathematischer Theorien}

\subsection{Definitionen}

\begin{frame}
\begin{mydef}
Seien $a,b \in \Z$.
Die Zahl $a$ heißt \textit{Teiler} von $b$ und $b$ heißt \textit{Vielfaches}
von $a$, wenn es ein $c \in \Z$ gibt mit $b = a \cdot c$.
Man schreibt dann $a \mid b$. Die Negation ist $a \nmid b$.
\end{mydef}

\begin{mydef}
Sei $n \in \Z$.
\begin{enumerate}
\item Die Zahl $n$ heißt \textit{gerade}, falls $2 \mid n$.
\item Falls $n$ nicht gerade ist, so heißt $n$ \textit{ungerade}.
\end{enumerate}
\end{mydef}

\begin{mydef}
Eine Zahl $p \in \N$ heißt \textit{Primzahl}, falls $p > 1$ ist und $1$ und
$p$ die einzigen natürlichen Zahlen sind die $p$ teilen.
Eine Zahl $n \in \N$, die keine Primzahl ist, heißt \textit{zusammengesetzt}.
\end{mydef}
\end{frame}

\subsection{Sätze und Beweise}

\begin{frame}
Unter einem mathematischen Satz (Lemma, Korollar, etc.) verstehen wir eine
nicht-triviale mathematische Aussage, für die ein gültiger Beweis vorliegt.

\begin{remark}
Zur Begriffsklärung:

\begin{itemize}
\item \textbf{Satz, Theorem}: Dies ist das typische Resultat einer Theorie.

\item \textbf{Hauptsatz}:
So wird ein besonders wichtiger Satz in einem Teilgebiet der Mathematik
genannt, beispielsweise der Hauptsatz der Differential- und Integralrechnung
aus der Analysis.
\end{itemize}
\end{remark}
\end{frame}


\begin{frame}
\begin{block}{Fortsetzung}
\begin{itemize}
\item \textbf{Lemma}:
Diese Bezeichnung wird in verschiedenen Zusammenhängen verwendet.
Zum einen bezeichnet es ein kleines, meist technisches Resultat, einen
\textit{Hilfssatz}, der zum Beweis eines wichtigen Satzes verwendet wird.
Zum anderen handelt es sich dabei um besonders wichtige
\textit{Schlüsselgedanken}, die in vielen Situationen nützlich sind.
Solche Lemmata werden dann auch häufig mit dem Namen ihres Erfinders bezeichnet
(z.B. Lemma von Pratt, Lemma von Urysohn, Lemma von Zorn, etc.).

\item \textbf{Proposition}:
Dies ist die lateinische Bezeichnung für Satz und wir manchmal synonym
verwendet, meist aber um ein Resultat zu bezeichnen, dessen Wichtigkeit
zwischen der eines Hilfssatzes und der eines Theorems liegt.

\item \textbf{Korollar}:
Dies ist die Bezeichnung für einen Satz, der aus einem anderen Satz durch
triviale oder sehr einfache Schlussweise folgt.
Manchmal ist es aber auch ein Spezialfall eines vorhergehenden Satzes, dem
besondere Aufmerksamkeit gebührt.
\end{itemize}
\end{block}
\end{frame}


\begin{frame}{Direkter Beweis}
Der direkte Beweis beweist eine Aussage durch schrittweises Folgen von Aussagen
auf Basis der gegebenen Voraussetzung.
Sei dazu $V$ die Voraussetzung und $B$ die zu zeigende Behauptung.
Unsere Aufgabe ist es also nun, geeignete Aussagen $A_1, A_2,...,A_n$ zu
finden, sodass schließlich gilt:
\[
  (V \Rightarrow A_1) \wedge (A_1 \Rightarrow A_2) \wedge ... \wedge
  (A_n \Rightarrow B)
\]
Ist uns dies gelungen, so haben wir gezeigt, dass aus der Voraussetzung $V$
stets auch die Behauptung $B$ folgt.
\end{frame}


\begin{frame}
Am besten veranschaulichen wir uns dies anhand eines einfachen Beispiels aus
der Zahlentheorie.

\begin{proposition}
Sei $a \in \Z$ gerade.
Dann ist auch $a^2 = a \cdot a$ gerade.
\end{proposition}

\begin{proof*}
Sei $a \in \Z$ gerade.
Nach Definition existiert dann $c \in \Z$ mit $a = 2 \cdot c$.
Somit folgt:
\[
  a^2 = a \cdot a = 2c \cdot 2c = 2(2c^2).
\]
Wegen $2c^2 \in \Z$ ist also auch $a^2$ gerade.
\hfill $\square$
\end{proof*}
\end{frame}


\begin{frame}
Allgemeiner lassen sich die folgenden Regeln zeigen:

\begin{theorem}
\begin{enumerate}
\item Für jedes $a \in \Z$ gilt:
\[
  a \mid 0, \quad 1 \mid a, \quad -1 \mid a,  \quad a \mid a, \quad -a \mid a,
  \quad a \mid -a .
\]

\item Für $a,b,c \in \Z$ gilt:
\[
  (a \mid b \ \wedge \ b \mid c) \Rightarrow a \mid c .
\]

\item Seien $a,b_1,...,b_n \in \Z$.
Gilt $a \mid b_i$ für alle $i \in \{1,...,n\}$,
so gilt für alle $x_1,...,x_n \in \Z$:
\[
  a \mid x_1 b_1 + ... + x_n b_n .
\]

\item Für $a,b,c,d \in \Z$ gilt:
\[
  (a \mid c \ \wedge  \ b \mid d) \Rightarrow ab \mid dc .
\]
\end{enumerate}
\end{theorem}
\end{frame}


\begin{frame}{Indirekter Beweis}
Eine weitere oft verwendete Beweistechnik ist die des
\textit{Indirekten Beweis}.
Diese beruht auf der folgenden logischen Äquivalenz:
Seien $V$ und $B$ zwei Aussagen, dann gilt:

\[
    (V \Rightarrow B) \iff (\neg B \Rightarrow \neg V).
\]

Diese Beweistechnik bietet sich in einigen Fällen an, da die rechte Implikation
manchmal einfacher zu zeigen ist als die linke.
Zusammengefasst ergibt sich das folgende Schema um eine Aussage der Form
$A \Rightarrow B$ zu zeigen:

\begin{enumerate}
\item Wir nehmen a n es gelte $\neg B$ (und bringen dies auch zum Ausdruck,
sodass auch der Leser oder Korrektor sieht, was wir vorhaben).

\item Aus der Aussage $\neg B$ und anderen uns zur Verfügung stehenden
Definitionen und Sätzen leiten wir $\neg V$ ab.

\item Wegen der oben beschrieben Äquivalenz gilt dann auch $V \Rightarrow B$.
\end{enumerate}
\end{frame}


\begin{frame}
Diese Vorgehensweise werden wir nun verwenden um den folgenden Satz zu beweisen.

\begin{theorem}
Sei $n \in \Z$, sodass $n^2$ gerade ist.
Dann ist auch $n$ gerade.
\end{theorem}

\begin{proof*}
Sei also $n \in \Z$ ungerade.
Wir zeigen nun, dass dann auch das Quadrat
von $n$ ungerade ist.
Da $n$ ungerade ist existiert $c \in \Z$ mit $n = 2c + 1$.
Somit erhalten wir durch Anwendung der ersten binomischen Formel:
\[
    n^2 = (2c +1)^2 = 4c^2 + 4c + 1 = 2(2c^2 + 2c) + 1 .
\]
Also ist $n^2$ ungerade und die Behauptung gezeigt.
\hfill $\square$
\end{proof*}
\end{frame}


\begin{frame}{Widerspruchsbeweis}
Der \textit{Widerspruchsbeweis} (auch bekannt als \textit{Reductio ad absurdum})
basiert auf der logischen Äquivalenz:
\[
(V \Rightarrow B) \iff \neg (V \wedge \neg B)
\]

Ein Beweis per Widerspruch verläuft also nach dem folgenden Schema:

\begin{enumerate}
\item Wir bringen zum Ausdruck, dass der Beweis per Widerspruch erfolgen soll.
Meist schreibt man einfach: \glqq Widerspruchsannahme: $\neg B$\grqq.
Auch hier ist es wichtig, dass man die Negation der Aussage $B$ richtig
formuliert.

\item Aus den Aussagen $V$ und $\neg B$ leiten wir nun eine Aussage ab, von der
wir bereits wissen, dass sie falsch ist.

\item Um zu zeigen, dass dies der gewünschte Widerspruch ist markieren wir die
Stelle durch einen Blitz oder durch das Wort \glqq Widerspruch\grqq.
\end{enumerate}
\end{frame}


\begin{frame}
\begin{lemma}
\begin{enumerate}
\item[(i)] Ist $b \in \Z \setminus \{0\}$ so gilt für jeden Teiler $a$ von $b$:
1 $\leq \lvert a \rvert \leq \lvert b \rvert$.

\item[(ii)] Die einzigen Teiler von $1$ sind $1$ und $-1$.

\item[(iii)] Für $a,b \in \Z$ gilt:
\[
  a \mid b \ \wedge \ b \mid a \iff b = a \text{ oder } b = -a.
\]
\end{enumerate}
\end{lemma}

\begin{proof*}
\textbf{zu (i):}
Seien $a,b \in \Z$ mit $b \neq 0$ und gelte $a \mid b$.
Dann existiert ein $n \in \Z$ mit $b = n \cdot a$ und somit gilt auch $n \neq 0$.
Die erste Ungleichung ist wegen $b \neq 0$ klar.
Angenommen es gilt $\lvert a \rvert > \lvert b \rvert$.
Dann folgt unter Verwendung elementarer Rechenregeln für die Betragsfunktion:
\[
  \lvert b \rvert = \lvert n a \rvert = \lvert n \rvert \lvert a \rvert
  \geq \lvert a \rvert > \lvert b \rvert,
\]
ein Widerspruch. Also gilt $1 \leq \lvert a \rvert \leq \lvert b \rvert$.\\
\textbf{zu (ii), (iii):} [Zur Übung].
\hfill $\square$
\end{proof*}
\end{frame}


\begin{frame}
\begin{lemma}
Sei $a \in \N$ mit $a > 1$.
Dann gibt es $r \in \N$ und Primzahlen $p_1,...,p_r$, sodass:
\begin{align}
    a = p_1 \cdot ... \cdot p_r \label{pfz}
\end{align}
Die Zerlegung (\ref{pfz}) wird auch als die \textit{Primfaktorzerlegung} von
$a$ bezeichnet.
\end{lemma}

\begin{proof*}
Angenommen die Behauptung des Lemmas ist falsch.
Dann gibt es insbesondere eine kleinste natürliche Zahl $a$ mit $a>1$,
für die keine derartige Zerlegung existiert.
Zunächst einmal halten wir fest, dass dann $a$ keine Primzahl sein kann,
denn sonst wäre $a$ ja trivialerweise ein Produkt von Primzahlen.
\end{proof*}
\end{frame}

\begin{frame}
\begin{block}{Fortsetzung}
Also gibt es eine Zahl $b \in \N \setminus \{1,a\}$ mit $b \mid a$.
Daher existiert nach Definition $c \in \N$ mit $a = b \cdot c$.
Nach Lemma 4.8 gilt ferner:
$b < a$ und $c < a$. Da $a$ die kleinste natürliche Zahl ist, die keine
derartige Zerlegung besitzt lassen sich $b$ und $c$ in Primfaktoren zerlegen:
\begin{align*}
    b &= p_1 \cdot \ldots \cdot p_s \\\
    c &= p_{s+1} \cdot \ldots \cdot p_k
\end{align*}
Somit folgt
\[
    a = b c = p_1 \ldots p_s p_{s+1} \ldots p_k
\]
im Widerspruch zur Voraussetzung.
Also besitzt jede natürliche Zahl eine Primfaktorzerlegung.
\hfill $\square$
\end{block}
\end{frame}


\begin{frame}
Auf Basis dieses Lemmas können wir nun den folgenden, auf Euklid
zurückgehenden Satz beweisen.
Auch diesen werden wir per Widerspruch beweisen.

\begin{theorem}
Es gibt unendlich viele Primzahlen.
\end{theorem}

\begin{proof*}
Die Negation von \textit{unendlich viele} ist \textit{endlich viele}.
Also nehmen wir an, dass es nur endlich viele Primzahlen gibt.
Sei also
\[
  P = \{p_1,\ldots,p_n\}
\]
die Menge aller Primzahlen.
Setze nun
\[
  a = p_1 \ldots p_n + 1.
\]
\end{proof*}
\end{frame}


\begin{frame}
\begin{block}{Fortsetzung}
Dann gilt offensichtlich $a>1$.
Nach Lemma 4.10 lässt sich $a$ also in Primfaktoren zerlegen.
Es gilt also $p_i \mid a$ für ein $i \in \{1,\ldots,n\}$.
Wegen $p_i \mid p_1 ... p_n$ gilt nach Satz 4.6:
$p_i \mid a - p_1 ... p_n = 1$.
Also $p_i \mid 1$.
Dies liefert $p_i = 1$.
Im Widerspruch dazu, dass $p_i$ eine Primzahl ist.
\hfill $\square$
\end{block}
\end{frame}


\begin{frame}
Ein weiteres klassisches Resultat, welches man per Widerspruch beweist, ist
folgender Satz, den wir schon zuvor als Beispiel gesehen haben.

\begin{theorem}
Die Zahl $\sqrt{2}$ ist irrational.
Das heißt falls $q \in \Q$, dann gilt $q \neq \sqrt{2}$.
\end{theorem}

\begin{proof*}
Angenommen $\sqrt{2}$ ist rational.
Dann existieren $p,q \in \N$ mit $\sqrt{2} = \frac{p}{q}$.
Ferner seien $p$ und $q$ teilerfremd.
Es gilt also insbesondere:
\[
  \left(\frac{p}{q}\right)^2 = 2 .
\]
Somit folgt durch Umstellen:
\[
  p^2 = 2q^2.
\]
\end{proof*}
\end{frame}


\begin{frame}
\begin{block}{Fortsetzung}
Also ist $p^2$ eine gerade Zahl.
Nach Satz 4.7 ist damit auch $p$ gerade, folglich existiert ein $n \in \N$
mit $p = 2n$.
Dies liefert wiederum:
\[
  2q^2 = p^2 = (2n)^2 = 4n^2.
\]
Also gilt insbesondere $q^2 = 2n^2$.
Daher ist auch $q$ durch zwei teilbar.
Im Widerspruch dazu, dass $p$ und $q$ teilerfremd sind.
Somit kann $\sqrt{2}$ keine rationale Zahl sein und die Behauptung ist gezeigt.
\hfill $\square$
\end{block}
\end{frame}

\section{Relationen und Abbildungen}

\subsection{Relationen}

\begin{frame}
\begin{mydef}
Seien $A,B$ zwei Mengen. Dann ist das \textit{kartesische Produkt} von $A$ und $B$
definiert durch:
\[
    A \times B := \{ (a,b) | a \in A \wedge b \in B \}.
\]
\end{mydef}

\begin{mydef}
Seien $A,B$ zwei Mengen und $A \times B$ das kartesische Produkt von
$A$ und $B$.
Eine Teilmenge $R \subseteq A \times B$ heißt dann auch eine (zweistellige)
\textit{Relation} (zwischen $X$ und $Y$).
Gilt $X = Y$ so spricht man auch von einer Relation \textit{auf} $X$.
Anstatt von $(x,y) \in R$ schreibt man auch $xRy$.
\end{mydef}
\end{frame}


\begin{frame}
\begin{example}
Die folgenden Mengen sind Relationen:

\begin{enumerate}
\item $R_1 := \{(a,b) | a \neq b \} \subseteq \Z \times \Z$.
\item $R_2 := \{(a,b) | a | b \} \subseteq \Z \times \Z$.
\item $R_3 := \{(a,b) | a \leq b\} \subseteq \R \times \R$.
\end{enumerate}
\end{example}
\end{frame}


\begin{frame}
Eine Relation kann verschiedene Eigenschaften haben:

\begin{mydef}
Sei $X$ eine Menge und $R \subseteq X \times X$ eine Relation.
$R$ heißt
\begin{enumerate}
\item \textit{reflexiv}, falls: $\forall x \in X\!:\ (x,x) \in R$,

\item \textit{symmetrisch}, falls:
$\forall x,y \in X\!:\ (x,y) \in R \Rightarrow (y,x) \in R$,

\item \textit{antisymmetrisch}, falls:
$\forall x,y \in X\!:\ (x,y) \in R \wedge (y,x) \in R \Rightarrow x = y$,

\item \textit{transitiv}, falls:
$\forall x,y,z \in X\!:\ (x,y) \in R \wedge (y,z) \in R
\Rightarrow (x,z) \in R$.
\end{enumerate}
\end{mydef}

\begin{example}
Die bereits aus der Schule bekannte Relation $\leq$ ist reflexiv,
antisymmetrisch und transitiv, aber nicht symmetrisch.
\end{example}
\end{frame}


\begin{frame}
\begin{mydef}
Eine Relation $R$ auf einer Menge $X$ wird als \textit{Äquivalenzrelation}
bezeichnet, falls $R$ reflexiv, symmetrisch und transitiv ist.
\end{mydef}

\begin{mydef}
Eine Relation $R$ auf einer Menge $X$ wird als \textit{partielle Ordnung}
bezeichnet, falls $R$ reflexiv, antisymmetrisch und transitiv ist.
\end{mydef}
\end{frame}


\begin{frame}
\begin{proposition}
Sei $X$ eine Menge.
Dann ist durch
\[
  R := \{ (A,B) | \  A \subseteq B \}
\]
eine partielle Ordnung auf $\mathcal{P}(X)$ definiert.
\end{proposition}

\begin{proposition}
Sei $m \in \Z$.
Dann ist durch
\[
    R_m := \{(a,b) | \ m \text{ teilt } (b-a) \}
\]
eine Äquivalenzrelation auf $\Z$ definiert.
\end{proposition}
\end{frame}


\subsection{Abbildungen}

\begin{frame}
\begin{mydef}
Seien $A,B$ nichtleere Mengen und $R$ eine Relation zwischen $A$ und $B$.
$R$ heißt

\begin{enumerate}
\item \textit{linkstotal}, falls:
$\forall a \in A \ \exists b \in B\!:\ (a,b) \in R$.

\item \textit{rechtseindeutig}, falls:
$\forall a \in A \forall b,c \in B\!:\ (a,b) \in R \wedge (a,c) \in R
\Rightarrow b = c $.
\end{enumerate}

Eine linkstotale und rechtseindeutige Relation wird als \textit{Abbildung} oder
auch als \textit{Funktion} bezeichnet.
\end{mydef}


\begin{remark}
Wenn man die Begriffe der Linkstotalität und der Rechtseindeutigkeit
zusammenfasst erhält man
\[
  \forall a \in A \  \exists ! b \in B\!:\ (a,b) \in R.
\]
Dies wird in der Literatur vereinzelt auch als \textit{Funktionseigenschaft}
bezeichnet.
\end{remark}
\end{frame}


\begin{frame}
\begin{remark}
Funktionen können natürlich mit der für Relationen kennengelernten Notation
aufgeschrieben werden, allerdings hat sich in der heutigen Zeit die folgende
Notation für eine Funktion $f$ von $A$ nach $B$ durchgesetzt:
\[
  f\!:\ A \to B, \quad x \mapsto f(x)
\]
Hierbei ist es wichtig anzumerken, dass jede Funktionsdefinition aus zwei
Bausteinen besteht:

\begin{enumerate}
\item Angabe von Definitions- und Wertebereich,
\item Angabe der Abbildungsvorschrift.
\end{enumerate}
\end{remark}
\end{frame}


\begin{frame}
\begin{example}
\begin{enumerate}
\item $f\!:\ \Z \to \Z,\ x \mapsto x^2$.

\item Sei $X$ eine nichtleere Menge. Dann heißt die Abbildung
\[
  id_X\!:\ X \to X,\ x \mapsto x
\]
die \textit{Identität} auf $X$.

\item Nur durch $x \mapsto x^2$ ist keine Funktion definiert, da
Definitions- und Wertebereich nicht spezifiziert sind.
\end{enumerate}
\end{example}
\end{frame}


\begin{frame}
Oftmals ist es notwendig, Funktionen auf spezielle Eigenschaften zu untersuchen.
Einige dieser Eigenschaften, die Ihnen im Laufe des Studiums noch häufiger
begegnen, werden im Folgenden vorgestellt.

\begin{mydef}
Seien $A,B$ nichtleere Mengen.
Eine Abbildung $f\!:\ A \to B$ heißt

\begin{enumerate}
\item \textit{injektiv}, falls:
\[
  \forall a_1,a_2 \in A\!:\ f(a_1)=f(a_2) \Rightarrow a_1 = a_2,
\]

\item \textit{surjektiv}, falls:
\[
  \forall b \in B \ \exists a \in A\!:\ f(a) = b,
\]

\item \textit{bijektiv}, falls $f$ surjektiv und injektiv ist.
\end{enumerate}
\end{mydef}
\end{frame}


\begin{frame}
\begin{example}
Betrachte die Funktion
\[
    f\!:\ \Z \to \Z, \ x \mapsto x^2.
\]
Dann ist $f$ nicht injektiv, denn es gilt: $f(-5) = 25 = f(5)$ und $-5 \neq 5$.
Zudem ist $f$ auch nicht surjektiv, denn es gilt $f(x) = x^2 \geq 0$ für alle
$x \in \Z$.
\end{example}
\end{frame}


\subsection{Bild und Urbild}

\begin{frame}
\begin{mydef}
Seien $X,Y$ nichtleere Mengen und sei $f\!:\ X \to Y$ eine Abbildung.

\begin{enumerate}
\item Sei $A \subseteq X$.
Dann heißt $f(A) := \{y \in Y | \exists x \in A\!:\ f(x) = y \}$
das \textit{Bild} von $A$ unter $f$.

\item Sei $B \subseteq Y$.
Dann heißt $f^{-1}(B) := \{x \in X | f(x) \in B \}$
das \textit{Urbild} von $B$ unter $f$.
\end{enumerate}
\end{mydef}
\end{frame}


\begin{frame}
\begin{example}
Betrachte die Abbildung
\[
  f\!:\ \R \to \R, \ x \mapsto -x,
\]
Sowie die Mengen $A = [0,2]$ und $B = \{1,3,8,9\}$.

Dann gilt:
\begin{align*}
  f(A)      &= [-2,0], \\\
    f^{-1}(A) &= \{-1, -3,-8,-9\}.
\end{align*}
\end{example}
\end{frame}


\begin{frame}
\begin{theorem}
Seien $X,Y$ nichtleere Mengen, $f\!:\ X \to Y$ eine Abbildung, $I$ eine Menge.
Weiter seien $A,B$ sowie $A_i$, $i \in I$, Teilmengen von $X$.
Dann gilt:

\begin{enumerate}
\item $f(A) = \emptyset \iff A = \emptyset$,
\item $A \subseteq B \Rightarrow f(A) \subseteq f(B)$,
\item $f(\bigcup_{i \in I} A_i) = \bigcup_{i \in I} f(A_i)$,
\item $f(\bigcap_{i \in I} A_i) \subseteq \bigcap_{i \in I} f(A_i)$, wobei im
Allgemeinen keine Gleichheit gilt.
\end{enumerate}
\end{theorem}
\end{frame}


\begin{frame}
\begin{theorem}
Seien $X,Y$ nichtleere Mengen, $f\!:\ X \to Y$ eine Abbildung.
Weiter seien $I$ eine Menge sowie $A,B$ und $A_i$, $i \in I$, Teilmengen von $Y$.
Dann gilt:

\begin{enumerate}
\item $f^{-1}(A) = \emptyset \iff A \cap f(X) = \emptyset$,
\item $A \subseteq B \Rightarrow f^{-1}(A) \subseteq f^{-1}(B)$,
\item $f^{-1}(\bigcup_{i \in I}A_i) = \bigcup_{i \in I}f^{-1}(A_i)$,
\item $f^{-1}(\bigcap_{i \in I}A_i) = \bigcap_{i \in I}f^{-1}(A_i)$,
\item $f^{-1}(Y \setminus A) = X \setminus f^{-1}(A)$.
\end{enumerate}
\end{theorem}
\end{frame}


\end{document}
