\documentclass[notheorems]{beamer}

% BEAMER SETTINGS
\usetheme{CambridgeUS}
\usecolortheme{dolphin}

\setbeamertemplate{theorems}[numbered]

\AtBeginSection[]
{
    \begin{frame}{Agenda}
        \tableofcontents[currentsection]
    \end{frame}
}

\AtBeginSubsection[]
{
    \begin{frame}{Agenda}
        \tableofcontents[currentsection,currentsubsection]
    \end{frame}
}


% IMPORTS
\usepackage[ngerman]{babel}
\usepackage[utf8x]{inputenc}
\usepackage{amsfonts}
\usepackage{amsmath}
\usepackage{amssymb}
\usepackage{amstext}
\usepackage{float}
\usepackage{hyperref}

% NEWCOMMANDS
% Syntax: \newcommand{\shortcut}{\wasrauskommensoll}

% Shortcuts für Zahlenbereiche:
\newcommand{\N}{\mathbb{N}}
\newcommand{\R}{\mathbb{R}}
\newcommand{\Z}{\mathbb{Z}}
\newcommand{\C}{\mathbb{C}}
\newcommand{\Q}{\mathbb{Q}}

% THEOREMSTYLES

\newtheorem{theorem}{Satz} % Sätze, sollte später evtl. noch Lemma und Korollar hinzufügen
\numberwithin{theorem}{section}

\newtheorem{mydef}[theorem]{Definition} % Definitionen
\numberwithin{theorem}{section}

\newtheorem{example}[theorem]{Beispiel} % Beispiele
\numberwithin{theorem}{section}

\newtheorem{proposition}[theorem]{Proposition} % Propositionen
\numberwithin{theorem}{section}

\newtheorem{remark}[theorem]{Bemerkung} % Bemerkungen
\numberwithin{theorem}{section}

\newtheorem{lemma}[theorem]{Lemma} % Lemmata
\numberwithin{theorem}{section}

\newtheorem*{proof*}{Beweis:}

% PDF-META
\hypersetup{pdftex,
            pdfauthor={Jonas Köppl, Maximilian Reif, Peter Mader},
            pdftitle={Brückenkurs Mathematik -- Beamer-Präsentation},
            pdfsubject={},
            pdfkeywords={},
            pdfproducer={},
            pdfcreator={},
            pdfpagemode=UseOutlines
}

\title{Brückenkurs Mathematik}
\author[]{Jonas Köppl, Maximilian Reif, Peter Mader}
\institute[]{Universität Passau}
\date{}

\begin{document}

\begin{frame}
    \titlepage
\end{frame}

\begin{frame}{Agenda}
    \tableofcontents
\end{frame}

\include{presentation-chapters/01-Einfuehrung}
\include{presentation-chapters/02-Grundlagen}
\include{presentation-chapters/03-Mengen}
\section{Aufbau mathematischer Theorien}

\subsection{Definitionen}

\begin{frame}
\begin{mydef}
Seien $a,b \in \Z$.
Die Zahl $a$ heißt \textit{Teiler} von $b$ und $b$ heißt \textit{Vielfaches}
von $a$, wenn es ein $c \in \Z$ gibt mit $b = a \cdot c$.
Man schreibt dann $a \mid b$. Die Negation ist $a \nmid b$.
\end{mydef}

\begin{mydef}
Sei $n \in \Z$.
\begin{enumerate}
\item Die Zahl $n$ heißt \textit{gerade}, falls $2 \mid n$.
\item Falls $n$ nicht gerade ist, so heißt $n$ \textit{ungerade}.
\end{enumerate}
\end{mydef}

\begin{mydef}
Eine Zahl $p \in \N$ heißt \textit{Primzahl}, falls $p > 1$ ist und $1$ und
$p$ die einzigen natürlichen Zahlen sind die $p$ teilen.
Eine Zahl $n \in \N$, die keine Primzahl ist, heißt \textit{zusammengesetzt}.
\end{mydef}
\end{frame}

\subsection{Sätze und Beweise}

\begin{frame}
Unter einem mathematischen Satz (Lemma, Korollar, etc.) verstehen wir eine
nicht-triviale mathematische Aussage, für die ein gültiger Beweis vorliegt.

\begin{remark}
Zur Begriffsklärung:

\begin{itemize}
\item \textbf{Satz, Theorem}: Dies ist das typische Resultat einer Theorie.

\item \textbf{Hauptsatz}:
So wird ein besonders wichtiger Satz in einem Teilgebiet der Mathematik
genannt, beispielsweise der Hauptsatz der Differential- und Integralrechnung
aus der Analysis.
\end{itemize}
\end{remark}
\end{frame}


\begin{frame}
\begin{block}{Fortsetzung}
\begin{itemize}
\item \textbf{Lemma}:
Diese Bezeichnung wird in verschiedenen Zusammenhängen verwendet.
Zum einen bezeichnet es ein kleines, meist technisches Resultat, einen
\textit{Hilfssatz}, der zum Beweis eines wichtigen Satzes verwendet wird.
Zum anderen handelt es sich dabei um besonders wichtige
\textit{Schlüsselgedanken}, die in vielen Situationen nützlich sind.
Solche Lemmata werden dann auch häufig mit dem Namen ihres Erfinders bezeichnet
(z.B. Lemma von Pratt, Lemma von Urysohn, Lemma von Zorn, etc.).

\item \textbf{Proposition}:
Dies ist die lateinische Bezeichnung für Satz und wir manchmal synonym
verwendet, meist aber um ein Resultat zu bezeichnen, dessen Wichtigkeit
zwischen der eines Hilfssatzes und der eines Theorems liegt.

\item \textbf{Korollar}:
Dies ist die Bezeichnung für einen Satz, der aus einem anderen Satz durch
triviale oder sehr einfache Schlussweise folgt.
Manchmal ist es aber auch ein Spezialfall eines vorhergehenden Satzes, dem
besondere Aufmerksamkeit gebührt.
\end{itemize}
\end{block}
\end{frame}


\begin{frame}{Direkter Beweis}
Der direkte Beweis beweist eine Aussage durch schrittweises Folgen von Aussagen
auf Basis der gegebenen Voraussetzung.
Sei dazu $V$ die Voraussetzung und $B$ die zu zeigende Behauptung.
Unsere Aufgabe ist es also nun, geeignete Aussagen $A_1, A_2,...,A_n$ zu
finden, sodass schließlich gilt:
\[
  (V \Rightarrow A_1) \wedge (A_1 \Rightarrow A_2) \wedge ... \wedge
  (A_n \Rightarrow B)
\]
Ist uns dies gelungen, so haben wir gezeigt, dass aus der Voraussetzung $V$
stets auch die Behauptung $B$ folgt.
\end{frame}


\begin{frame}
Am besten veranschaulichen wir uns dies anhand eines einfachen Beispiels aus
der Zahlentheorie.

\begin{proposition}
Sei $a \in \Z$ gerade.
Dann ist auch $a^2 = a \cdot a$ gerade.
\end{proposition}

\begin{proof*}
Sei $a \in \Z$ gerade.
Nach Definition existiert dann $c \in \Z$ mit $a = 2 \cdot c$.
Somit folgt:
\[
  a^2 = a \cdot a = 2c \cdot 2c = 2(2c^2).
\]
Wegen $2c^2 \in \Z$ ist also auch $a^2$ gerade.
\hfill $\square$
\end{proof*}
\end{frame}


\begin{frame}
Allgemeiner lassen sich die folgenden Regeln zeigen:

\begin{theorem}
\begin{enumerate}
\item Für jedes $a \in \Z$ gilt:
\[
  a \mid 0, \quad 1 \mid a, \quad -1 \mid a,  \quad a \mid a, \quad -a \mid a,
  \quad a \mid -a .
\]

\item Für $a,b,c \in \Z$ gilt:
\[
  (a \mid b \ \wedge \ b \mid c) \Rightarrow a \mid c .
\]

\item Seien $a,b_1,...,b_n \in \Z$.
Gilt $a \mid b_i$ für alle $i \in \{1,...,n\}$,
so gilt für alle $x_1,...,x_n \in \Z$:
\[
  a \mid x_1 b_1 + ... + x_n b_n .
\]

\item Für $a,b,c,d \in \Z$ gilt:
\[
  (a \mid c \ \wedge  \ b \mid d) \Rightarrow ab \mid dc .
\]
\end{enumerate}
\end{theorem}
\end{frame}


\begin{frame}{Indirekter Beweis}
Eine weitere oft verwendete Beweistechnik ist die des
\textit{Indirekten Beweis}.
Diese beruht auf der folgenden logischen Äquivalenz:
Seien $V$ und $B$ zwei Aussagen, dann gilt:

\[
    (V \Rightarrow B) \iff (\neg B \Rightarrow \neg V).
\]

Diese Beweistechnik bietet sich in einigen Fällen an, da die rechte Implikation
manchmal einfacher zu zeigen ist als die linke.
Zusammengefasst ergibt sich das folgende Schema um eine Aussage der Form
$A \Rightarrow B$ zu zeigen:

\begin{enumerate}
\item Wir nehmen a n es gelte $\neg B$ (und bringen dies auch zum Ausdruck,
sodass auch der Leser oder Korrektor sieht, was wir vorhaben).

\item Aus der Aussage $\neg B$ und anderen uns zur Verfügung stehenden
Definitionen und Sätzen leiten wir $\neg V$ ab.

\item Wegen der oben beschrieben Äquivalenz gilt dann auch $V \Rightarrow B$.
\end{enumerate}
\end{frame}


\begin{frame}
Diese Vorgehensweise werden wir nun verwenden um den folgenden Satz zu beweisen.

\begin{theorem}
Sei $n \in \Z$, sodass $n^2$ gerade ist.
Dann ist auch $n$ gerade.
\end{theorem}

\begin{proof*}
Sei also $n \in \Z$ ungerade.
Wir zeigen nun, dass dann auch das Quadrat
von $n$ ungerade ist.
Da $n$ ungerade ist existiert $c \in \Z$ mit $n = 2c + 1$.
Somit erhalten wir durch Anwendung der ersten binomischen Formel:
\[
    n^2 = (2c +1)^2 = 4c^2 + 4c + 1 = 2(2c^2 + 2c) + 1 .
\]
Also ist $n^2$ ungerade und die Behauptung gezeigt.
\hfill $\square$
\end{proof*}
\end{frame}


\begin{frame}{Widerspruchsbeweis}
Der \textit{Widerspruchsbeweis} (auch bekannt als \textit{Reductio ad absurdum})
basiert auf der logischen Äquivalenz:
\[
(V \Rightarrow B) \iff \neg (V \wedge \neg B)
\]

Ein Beweis per Widerspruch verläuft also nach dem folgenden Schema:

\begin{enumerate}
\item Wir bringen zum Ausdruck, dass der Beweis per Widerspruch erfolgen soll.
Meist schreibt man einfach: \glqq Widerspruchsannahme: $\neg B$\grqq.
Auch hier ist es wichtig, dass man die Negation der Aussage $B$ richtig
formuliert.

\item Aus den Aussagen $V$ und $\neg B$ leiten wir nun eine Aussage ab, von der
wir bereits wissen, dass sie falsch ist.

\item Um zu zeigen, dass dies der gewünschte Widerspruch ist markieren wir die
Stelle durch einen Blitz oder durch das Wort \glqq Widerspruch\grqq.
\end{enumerate}
\end{frame}


\begin{frame}
\begin{lemma}
\begin{enumerate}
\item[(i)] Ist $b \in \Z \setminus \{0\}$ so gilt für jeden Teiler $a$ von $b$:
1 $\leq \lvert a \rvert \leq \lvert b \rvert$.

\item[(ii)] Die einzigen Teiler von $1$ sind $1$ und $-1$.

\item[(iii)] Für $a,b \in \Z$ gilt:
\[
  a \mid b \ \wedge \ b \mid a \iff b = a \text{ oder } b = -a.
\]
\end{enumerate}
\end{lemma}

\begin{proof*}
\textbf{zu (i):}
Seien $a,b \in \Z$ mit $b \neq 0$ und gelte $a \mid b$.
Dann existiert ein $n \in \Z$ mit $b = n \cdot a$ und somit gilt auch $n \neq 0$.
Die erste Ungleichung ist wegen $b \neq 0$ klar.
Angenommen es gilt $\lvert a \rvert > \lvert b \rvert$.
Dann folgt unter Verwendung elementarer Rechenregeln für die Betragsfunktion:
\[
  \lvert b \rvert = \lvert n a \rvert = \lvert n \rvert \lvert a \rvert
  \geq \lvert a \rvert > \lvert b \rvert,
\]
ein Widerspruch. Also gilt $1 \leq \lvert a \rvert \leq \lvert b \rvert$.\\
\textbf{zu (ii), (iii):} [Zur Übung].
\hfill $\square$
\end{proof*}
\end{frame}


\begin{frame}
\begin{lemma}
Sei $a \in \N$ mit $a > 1$.
Dann gibt es $r \in \N$ und Primzahlen $p_1,...,p_r$, sodass:
\begin{align}
    a = p_1 \cdot ... \cdot p_r \label{pfz}
\end{align}
Die Zerlegung (\ref{pfz}) wird auch als die \textit{Primfaktorzerlegung} von
$a$ bezeichnet.
\end{lemma}

\begin{proof*}
Angenommen die Behauptung des Lemmas ist falsch.
Dann gibt es insbesondere eine kleinste natürliche Zahl $a$ mit $a>1$,
für die keine derartige Zerlegung existiert.
Zunächst einmal halten wir fest, dass dann $a$ keine Primzahl sein kann,
denn sonst wäre $a$ ja trivialerweise ein Produkt von Primzahlen.
\end{proof*}
\end{frame}

\begin{frame}
\begin{block}{Fortsetzung}
Also gibt es eine Zahl $b \in \N \setminus \{1,a\}$ mit $b \mid a$.
Daher existiert nach Definition $c \in \N$ mit $a = b \cdot c$.
Nach Lemma 4.8 gilt ferner:
$b < a$ und $c < a$. Da $a$ die kleinste natürliche Zahl ist, die keine
derartige Zerlegung besitzt lassen sich $b$ und $c$ in Primfaktoren zerlegen:
\begin{align*}
    b &= p_1 \cdot \ldots \cdot p_s \\\
    c &= p_{s+1} \cdot \ldots \cdot p_k
\end{align*}
Somit folgt
\[
    a = b c = p_1 \ldots p_s p_{s+1} \ldots p_k
\]
im Widerspruch zur Voraussetzung.
Also besitzt jede natürliche Zahl eine Primfaktorzerlegung.
\hfill $\square$
\end{block}
\end{frame}


\begin{frame}
Auf Basis dieses Lemmas können wir nun den folgenden, auf Euklid
zurückgehenden Satz beweisen.
Auch diesen werden wir per Widerspruch beweisen.

\begin{theorem}
Es gibt unendlich viele Primzahlen.
\end{theorem}

\begin{proof*}
Die Negation von \textit{unendlich viele} ist \textit{endlich viele}.
Also nehmen wir an, dass es nur endlich viele Primzahlen gibt.
Sei also
\[
  P = \{p_1,\ldots,p_n\}
\]
die Menge aller Primzahlen.
Setze nun
\[
  a = p_1 \ldots p_n + 1.
\]
\end{proof*}
\end{frame}


\begin{frame}
\begin{block}{Fortsetzung}
Dann gilt offensichtlich $a>1$.
Nach Lemma 4.10 lässt sich $a$ also in Primfaktoren zerlegen.
Es gilt also $p_i \mid a$ für ein $i \in \{1,\ldots,n\}$.
Wegen $p_i \mid p_1 ... p_n$ gilt nach Satz 4.6:
$p_i \mid a - p_1 ... p_n = 1$.
Also $p_i \mid 1$.
Dies liefert $p_i = 1$.
Im Widerspruch dazu, dass $p_i$ eine Primzahl ist.
\hfill $\square$
\end{block}
\end{frame}


\begin{frame}
Ein weiteres klassisches Resultat, welches man per Widerspruch beweist, ist
folgender Satz, den wir schon zuvor als Beispiel gesehen haben.

\begin{theorem}
Die Zahl $\sqrt{2}$ ist irrational.
Das heißt falls $q \in \Q$, dann gilt $q \neq \sqrt{2}$.
\end{theorem}

\begin{proof*}
Angenommen $\sqrt{2}$ ist rational.
Dann existieren $p,q \in \N$ mit $\sqrt{2} = \frac{p}{q}$.
Ferner seien $p$ und $q$ teilerfremd.
Es gilt also insbesondere:
\[
  \left(\frac{p}{q}\right)^2 = 2 .
\]
Somit folgt durch Umstellen:
\[
  p^2 = 2q^2.
\]
\end{proof*}
\end{frame}


\begin{frame}
\begin{block}{Fortsetzung}
Also ist $p^2$ eine gerade Zahl.
Nach Satz 4.7 ist damit auch $p$ gerade, folglich existiert ein $n \in \N$
mit $p = 2n$.
Dies liefert wiederum:
\[
  2q^2 = p^2 = (2n)^2 = 4n^2.
\]
Also gilt insbesondere $q^2 = 2n^2$.
Daher ist auch $q$ durch zwei teilbar.
Im Widerspruch dazu, dass $p$ und $q$ teilerfremd sind.
Somit kann $\sqrt{2}$ keine rationale Zahl sein und die Behauptung ist gezeigt.
\hfill $\square$
\end{block}
\end{frame}

\include{presentation-chapters/05-Relationen}

\end{document}
